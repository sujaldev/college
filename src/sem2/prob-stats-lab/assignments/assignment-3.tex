\startassignment{Measures of Central Tendency}

%----------------------------------------------------------------------------------------------------------------------%
\firstquestion{Caculate the mean of a vector.}
\begin{code}
    {Program}{r}
a <- c(28, 47, 29, 74, 47, 67, 83, 58, 93, 10)
mean(a)
\end{code}
\begin{code}
    {Output}{text}
[1] 53.6
\end{code}
%----------------------------------------------------------------------------------------------------------------------%

%----------------------------------------------------------------------------------------------------------------------%
\question{Caculate the mean and median of a vector $x$.}
\begin{code}
    {Program}{r}
x <- c(28, 47, 29, 74, 47, 67, 83, 58, 93, 10)
mean(x)
median(x)
\end{code}
\begin{code}
    {Output}{text}
[1] 53.6
[1] 52.5
\end{code}
%----------------------------------------------------------------------------------------------------------------------%

%----------------------------------------------------------------------------------------------------------------------%
\question{Caculate the geometric mean of a vector.}
\begin{code}
    {Program}{r}
x <- c(4, 12, 36, 108, 324, 972, 2916)
sqrt(x[1] * x[length(x)])
# OR
exp(mean(log(x)))
\end{code}
\begin{code}
    {Output}{text}
[1] 108
[1] 108
\end{code}
%----------------------------------------------------------------------------------------------------------------------%

%----------------------------------------------------------------------------------------------------------------------%
\question{Caculate the harmonic mean of a vector.}
\begin{code}
    {Program}{r}
library("psych")
x <- c(4, 12, 36, 108, 324, 972, 2916)
harmonic.mean(x)
\end{code}
\begin{code}
    {Output}{text}
[1] 18.67521
\end{code}
\newpage
%----------------------------------------------------------------------------------------------------------------------%

%----------------------------------------------------------------------------------------------------------------------%
\firstquestion{Caculate the arithmetic, geometric and harmonic mean of a vector.}
\begin{code}
    {Program}{r}
library("psych")
x <- c(4, 12, 36, 108, 324, 972, 2916)
mean(x)
geometric.mean(x)
harmonic.mean(x)
harmonic.mean(x)
\end{code}
\begin{code}
    {Output}{text}
[1] 624.5714
[1] 108
[1] 18.67521
\end{code}
%----------------------------------------------------------------------------------------------------------------------%

%----------------------------------------------------------------------------------------------------------------------%
\question{Caculate the first, second and third quartile of a vector.}
\begin{code}
    {Program}{r}
x <- c(4, 12, 36, 108, 324, 972, 2916, 8748)
quantile(x, 1 / 4)
quantile(x, 2 / 4)
quantile(x, 3 / 4)
\end{code}
\begin{code}
    {Output}{text}
25%
 30
50%
216
 75%
1458
\end{code}
%----------------------------------------------------------------------------------------------------------------------%

%----------------------------------------------------------------------------------------------------------------------%
\question{Caculate the third, fifth and sixth decile of a vector.}
\begin{code}
    {Program}{r}
x <- c(4, 12, 36, 108, 324, 972, 2916, 8748)
quantile(x, 3 / 10)
quantile(x, 5 / 10)
quantile(x, 6 / 10)
\end{code}
\begin{code}
    {Output}{text}
 30%
43.2
50%
216
  60%
453.6
\end{code}
\newpage
\pagestyle{fancy}
%----------------------------------------------------------------------------------------------------------------------%

%----------------------------------------------------------------------------------------------------------------------%
\firstquestion{Caculate the mean and median of a vector and ignore NA values.}
\begin{code}
    {Program}{r}
x <- c(4, 12, NA, 108, 324, NA, 2916, 8748)
mean(x, na.rm = TRUE)
median(x, na.rm = TRUE)
\end{code}
\begin{code}
    {Output}{text}
[1] 2018.667
[1] 216
\end{code}
%----------------------------------------------------------------------------------------------------------------------%

%----------------------------------------------------------------------------------------------------------------------%
\question{Caculate the arithmetic, geometric and harmonic mean with NA removal.}
\begin{code}
    {Program}{r}
library("psych")
x <- c(4, 12, 36, 108, 324, 972, 2916)
mean(x, na.rm = TRUE)
geometric.mean(x, na.rm = TRUE)
harmonic.mean(x, na.rm = TRUE)
\end{code}
\begin{code}
    {Output}{text}
[1] 624.5714
[1] 108
[1] 18.67521
\end{code}
%----------------------------------------------------------------------------------------------------------------------%

%----------------------------------------------------------------------------------------------------------------------%
\question{Find the first and third quartile of a vector with NA removal.}
\begin{code}
    {Program}{r}
x <- c(4, 12, NA, 108, 324, NA, 2916, 8748)
quantile(x, 1 / 4, na.rm = TRUE)
quantile(x, 3 / 4, na.rm = TRUE)
\end{code}
\begin{code}
    {Output}{text}
25%
 36
 75%
2268
\end{code}
%----------------------------------------------------------------------------------------------------------------------%
