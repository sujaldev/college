\startassignment{Probability and Probability Distributions}
\graphicspath{{../figures/assignment5/}}

%----------------------------------------------------------------------------------------------------------------------%
\firstquestion{A box contains 4 red, 3 white and 2 blue balls. Three balls are drawn at random. Find out the number of
ways of selecting the balls of different colours.}
\begin{code}
    {Program}{r}
#    RED           WHITE          BLUE
choose(4, 1) * choose(3, 1) * choose(2, 1)
\end{code}
\begin{code}
    {Output}{text}
[1] 24
\end{code}
%----------------------------------------------------------------------------------------------------------------------%

%----------------------------------------------------------------------------------------------------------------------%
\question{Find the probability of picking two Ace cards from a well shuffled pack of 52 playing cards.}
\begin{code}
    {Program}{r}
choose(4, 2) / choose(52, 2)
\end{code}
\begin{code}
    {Output}{text}
[1] 0.004524887
\end{code}
%----------------------------------------------------------------------------------------------------------------------%

%----------------------------------------------------------------------------------------------------------------------%
\question{If $\mathbf{X\sim Bino(10, 0.6)}$. Find:%
    \begin{enumerate}[label=(\alph*)]%
        \item $\mathbf{P(X=0)}$
        \item $\mathbf{P(X=2)}$
        \item $\mathbf{P(X\leq3)}$
    \end{enumerate}}~
\begin{code}
    {Program}{r}
al <- dbinom(0, 10, 0.6)
al
bl <- dbinom(2, 10, 0.6)
bl
cl <- dbinom(3, 10, 0.6)
cl
dl <- dbinom(5, 10, 0.6)
dl
\end{code}
\begin{code}
    {Output}{text}
[1] 0.0001048576
[1] 0.01061683
[1] 0.04246733
[1] 0.2006581
\end{code}
\newpage
\pagestyle{fancy}
%----------------------------------------------------------------------------------------------------------------------%

%----------------------------------------------------------------------------------------------------------------------%
\firstquestion{If $\mathbf{X\sim P(3.2)}$. Find:%
    \begin{enumerate}[label=(\alph*)]%
        \item $\mathbf{P(X = 0)}$
        \item $\mathbf{P(X = 3)}$
        \item $\mathbf{P(X = 5)}$
        \item $\mathbf{P(X \leq 1)}$
        \item $\mathbf{P(X > 3)}$
        \item $\mathbf{P(X \geq 5)}$
    \end{enumerate}}~
\begin{code}
    {Program}{r}
al <- dpois(0, 3.2)
al
bl <- dpois(3, 3.2)
bl
cl <- dpois(5, 3.2)
cl
dl <- ppois(10, 3.2)
dl
el <- 1 - ppois(3, 3.2)
el
fl <- 1 - ppois(5, 3.2)
fl
\end{code}
\begin{code}
    {Output}{text}
[1] 0.0407622
[1] 0.222616
[1] 0.1139794
[1] 0.9995028
[1] 0.3974803
[1] 0.1054081
\end{code}
%----------------------------------------------------------------------------------------------------------------------%

%----------------------------------------------------------------------------------------------------------------------%
\question{Fit the Poisson distribution to the following data with respect to the number of red blood corpuscles (x)
    per cell - x:\\[10pt]
    \begin{tblr}{colspec={|Q[1,c,m]|Q[1,c,m]|Q[1,c,m]|Q[1,c,m]|Q[1,c,m]|Q[1,c,m]|},hlines,vlines}%
        0   & 1   & 2  & 3  & 4 & 5 \\
        142 & 156 & 69 & 27 & 5 & 1 \\
    \end{tblr}}\\[10pt]
\begin{code}
    {Program}{r}
x <- 0:5; f <- c(142, 156, 69, 27, 5, 1)
m <- sum(x * f) / sum(f)
px <- dpois(x, m); px <- round(px, 4)
ef <- sum(f) * px
ef1 <- round(ef, 0)
d <- data.frame(x, f, "expected frequency" = ef)
d
plot(f, ef1, pch = "x"); abline(0, 1)
\end{code}
\begin{code}
    {Output}{text}
  x   f expected.frequency
1 0 142             147.16
2 1 156             147.16
3 2  69              73.56
4 3  27              24.52
5 4   5               6.12
6 5   1               1.24
\end{code}
\figurefigure{q5}
%----------------------------------------------------------------------------------------------------------------------%

%----------------------------------------------------------------------------------------------------------------------%
\question{Plot the probability mass function (PMF) and distribution function for the following random variables
    $\mathbf{X \sim P(2.6)}$.}
\begin{code}
    {Program}{r}
m <- 2.6
x <- 0:10
p <- dpois(x, m)
d <- data.frame(x, p)
d
plot(x, p, "h")

cp <- ppois(x, m)
cp1 <- round(cp, 4)
d1 <- data.frame(x, cp1)
plot(x, cp1, "s")
\end{code}
\newpage
\begin{code}
    {Output}{text}
    x            p
1   0 0.0742735782
2   1 0.1931113034
3   2 0.2510446944
4   3 0.2175720684
5   4 0.1414218445
6   5 0.0735393591
7   6 0.0318670556
8   7 0.0118363349
9   8 0.0038468089
10  9 0.0011113003
11 10 0.0002889381
\end{code}
\figurefigure{q6}
\figurefigure{q6-2}
%----------------------------------------------------------------------------------------------------------------------%

%----------------------------------------------------------------------------------------------------------------------%
\question{Plot the probability mass function (PMF) and distribution function for the following random variables
    $\mathbf{X \sim Bino(8,0.65)}$.}
\begin{code}
    {Program}{r}
n <- 8
p <- 0.65
x <- 0:n
bp <- dbinom(x, n, p)
d <- data.frame(x, "probabilities" = bp)
d
plot(x, bp, "h")
\end{code}
\begin{code}
    {Output}{text}
  x probabilities
1 0  0.0002251875
2 1  0.0033456434
3 2  0.0217466823
4 3  0.0807733916
5 4  0.1875096590
6 5  0.2785857791
7 6  0.2586867948
8 7  0.1372623809
9 8  0.0318644813
\end{code}
\figurefigure{q7}
\newpage
\figurefigure{q7-2}
\figurefigure{q7-3}
%----------------------------------------------------------------------------------------------------------------------%

%----------------------------------------------------------------------------------------------------------------------%
\question{Plot the probability mass function (PMF) and distribution function for the following random variables
    $\mathbf{X \sim HyperGeo(N=50,~M=10,~n=7)}$.}
\begin{code}
    {Program}{r}
N <- 50; M <- 10; n <- 7; x <- 0:n
hp <- dhyper(x, M, N - M, n)
d <- data.frame(x, hp); d
plot(x, hp, "h")

cp <- phyper(x, M, N - M, n)
cp1 <- round(cp, 4)
di <- data.frame(x, cp1); di
plot(x, cp1, "s")
\end{code}
\begin{code}
    {Output}{text}
  x           hp
1 0 1.866514e-01
2 1 3.842822e-01
3 2 2.964463e-01
4 3 1.097949e-01
5 4 2.077201e-02
6 5 1.967875e-03
7 6 8.409722e-05
8 7 1.201389e-06
  x    cp1
1 0 0.1867
2 1 0.5709
3 2 0.8674
4 3 0.9772
5 4 0.9979
6 5 0.9999
7 6 1.0000
8 7 1.000
\end{code}
\figurefigure[0.38]{q8}
\figurefigure[0.38]{q8-2}
\newpage
%----------------------------------------------------------------------------------------------------------------------%

%----------------------------------------------------------------------------------------------------------------------%
\question{Fit a normal distribution to the following data of height (in cms) of 200 Indian adult males:\\[10pt]%
    \begin{tblr}{colspec={|Q[1,c,m]|Q[1,c,m]|Q[1,c,m]|Q[1,c,m]|Q[1,c,m]|Q[1,c,m]|},hlines,vlines}%
        Height (cm)   & 144--150 & 150--156 & 156--162 & 162--168 & 168--174 & 174--180 & 180--186 \\
        No. of Adults & 3        & 12       & 23       & 52       & 61       & 39       & 10       \\
    \end{tblr}\\[10pt]}
\begin{code}
    {Program}{r}
li <- seq(144, 180, 6); ul <- seq(150, 186, 6)
f <- c(3, 12, 23, 52, 61, 39, 10)
x <- (li + ul) / 2; n <- sum(f); k <- length(f)
m <- sum(f * x) / n; v <- sum(f * (x - m)^2) / n; sd <- sqrt(v)
l1 <- c(-9999, li, 186)
cp <- pnorm(l1, m, sd)
p <- diff(cp)
p <- c(p, 1 - cp[k + 2])
ul <- c(144, ul, 9999); f <- c(0, f, 0)
ef <- round(n * p, 0)
d <- data.frame(
  "Lower Limit" = l1, "Upper Limit" = ul, "Obs. freq" = f,
  "prob" = p, "cumprob" = cp, "expfreq" = ef
)
d
plot(f, ef, xlab = "obs.freq", ylab = "exp.freq", "p"); abline(0, 1)
\end{code}
\begin{code}
    {Output}{text}
 Lower.Limit Upper.Limit Obs..freq         prob      cumprob expfreq
1       -9999         144         0 0.0009277682 0.0000000000       0
2         144         150         3 0.0085408285 0.0009277682       2
3         150         156        12 0.0474590553 0.0094685967       9
4         156         162        23 0.1504843558 0.0569276520      30
5         162         168        52 0.2727415211 0.2074120077      55
6         168         174        61 0.2828190953 0.4801535289      57
7         174         180        39 0.1677990586 0.7629726242      34
8         180         186        10 0.0569156032 0.9307716828      11
9         186        9999         0 0.0123127140 0.9876872860       2
\end{code}
\newpage
\figurefigure{q9}
%----------------------------------------------------------------------------------------------------------------------%

%----------------------------------------------------------------------------------------------------------------------%
\question{Let $\mathbf{X \sim N(5,40)}$. Find $\mathbf{P(X \leq 60), P(X \geq 100), P(10 \leq x \leq 20)}$ and
    $\mathbf{P(X \leq k) = 0.293}$.}
\begin{code}
    {Program}{r}
mu <- 50; sd <- sqrt(40)
pl <- pnorm(60, mu, sd)
pl
p2 <- 1 - pnorm(100, mu, sd)
p2
p3 <- pnorm(20, mu, sd)
p3
p4 <- qnorm(0.293, mu, sd)
p4
\end{code}
\begin{code}
    {Output}{text}
[1] 0.9430769
[1] 1.332268e-15
[1] 1.050718e-06
[1] 46.55538
\end{code}
%----------------------------------------------------------------------------------------------------------------------%
