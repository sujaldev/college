%! Author = Sujal Singh
%! Date = 5/8/24

% Preamble
%! suppress = FileNotFound
\documentclass[11pt]{ipu-python}
\usepackage[
    pdftitle={Programming in Python Practical File},
    pdfsubject={Programming in Python Practical File},
    pdfauthor={Sujal Singh},
    pdfdisplaydoctitle,
    hidelinks,
]{hyperref}

% Packages
\usepackage{amsmath}

% Document
\makeindex
\begin{document}
    \maketitle
    % ---------------------------------------------------------------------------------------------------------------- %

    % ---------------------------------------------------------------------------------------------------------------- %
    \question{%
        Write a program to perform string maniuplation operations using set of pre-defined functions such as:
        \begin{itemize}
            \item find()
            \item upper()
            \item len()
            \item max() and min()
            \item fetching specific content from string
        \end{itemize}}\vspace*{10pt}
    \begin{code}
        {Program}{python}
string = "Hello, World!"

print(
    f'string: "{string}"\n',
    f'find("llo"): {string.find("llo")}',
    f'upper(): {string.upper()}',
    f'len(): {len(string)}',
    f'max(): "{max(string)}"',
    f'min(): "{min(string)}"',
    f'fetching "llo": "{string[(start := string.find("llo")):start + len("llo")]}"',
    sep="\n"
)
    \end{code}
    \begin{code}
        {Output}{text}
string: "Hello, World!"

find("llo"): 2
upper(): HELLO, WORLD!
len(): 13
max(): "r"
min(): " "
fetching "llo": "llo"
    \end{code}
    % ---------------------------------------------------------------------------------------------------------------- %

    % ---------------------------------------------------------------------------------------------------------------- %
    \\~\vfill%
    \question{%
        Write a program to test and check the mathematical functions such as:
        \begin{itemize}
            \item ciel()
            \item sqrt()
            \item pow()
            \item factorial()
        \end{itemize}}\vfill
    \newpage
    \begin{code}
        {Program}{python}
from math import ceil, sqrt, pow, factorial

num = 3.14

print(
    f"num: {num}\n",
    f"ceil(num): {ceil(num)}",
    f"sqrt(num): {sqrt(num)}",
    f"pow(num, 2): {pow(num, 2)}",
    f"factorial(5): {factorial(5)}",
    sep="\n"
)
    \end{code}
    \begin{code}
        {Output}{text}
num: 3.14

ceil(num): 4
sqrt(num): 1.772004514666935
pow(num, 2): 9.8596
factorial(5): 120
    \end{code}
    % ---------------------------------------------------------------------------------------------------------------- %

    % ---------------------------------------------------------------------------------------------------------------- %
    \\~\\%
    \question{Write a program that receives a number as input from user and returns if it's an odd or even number.}
    \begin{code}
        {Program}{python}
num = int(input("Enter an integer: "))
print(f"{num} is {'even' if num % 2 == 0 else 'odd'}.")
    \end{code}
    \begin{code}
        {Output}{text}
Enter an integer: 42
42 is even.
    \end{code}
    % ---------------------------------------------------------------------------------------------------------------- %

    % ---------------------------------------------------------------------------------------------------------------- %
    \\~\\
    \question{Write a program that receives input from the user to calculate the area of a triangle.}
    \begin{code}
        {Program}{python}
base = int(input("Enter base length of triangle(m): "))
height = int(input("Enter height of triangle(m): "))
print("area =", 0.5 * base * height)
    \end{code}
    \begin{code}
        {Output}{text}
Enter base length of triangle(m): 5
Enter height of triangle(m): 4
area = 10.0
    \end{code}
    \newpage
    % ---------------------------------------------------------------------------------------------------------------- %

    % ---------------------------------------------------------------------------------------------------------------- %
    \question{Write a program that receives input from the user to calculate the area of a square.}
    \begin{code}
        {Program}{python}
side = int(input("Enter side length of square(m): "))
print(f"Area of square of side length {side} is {side * side}.")
    \end{code}
    \begin{code}
        {Output}{text}
Enter side length of square(m): 4
Area of square of side length 4 is 16.
    \end{code}
    % ---------------------------------------------------------------------------------------------------------------- %

    % ---------------------------------------------------------------------------------------------------------------- %
    \\~\\
    \question{Write a program that receives input from the user to calculate the area of a rectangle.}
    \begin{code}
        {Program}{python}
length = int(input("Enter length: "))
breadth = int(input("Enter breadth: "))
print(f"Area of rectangle having length {length} and breadth {breadth} is {length * breadth}.")
    \end{code}
    \begin{code}
        {Output}{text}
Enter length: 5
Enter breadth: 4
Area of rectangle having length 5 and breadth 4 is 20.
    \end{code}
    % ---------------------------------------------------------------------------------------------------------------- %

    % ---------------------------------------------------------------------------------------------------------------- %
    \\~\\
    \question{Write a program to check if the input string is a palindrome or not.}
    \begin{code}
        {Program}{python}
original = input("Enter a string: ")
print("Given string is", "a" if original == original[::-1] else "not a", "palindrome.")
    \end{code}
    \begin{code}
        {Output}{text}
Enter a string: ehehe
Given string is a palindrome.
    \end{code}
    % ---------------------------------------------------------------------------------------------------------------- %

    % ---------------------------------------------------------------------------------------------------------------- %
    \question{Write a program that receives marks of a student for a subject as input and assigns a grade A$\|$B$\|$C%
        $\|$D$\|$E$\|$F.}
    \begin{code}
        {Program}{python}
offset = 9 - int(input("Enter marks for subject (0 - 100): ")) // 10
# Makes negative offset values equal zero
positive_offset = (abs(offset) + offset) // 2
print(chr(65 + positive_offset) if offset <= 5 else "F")
    \end{code}
    \begin{code}
        {Output}{text}
Enter marks for subject (0 - 100): 90
A
    \end{code}
    % ---------------------------------------------------------------------------------------------------------------- %

    % ---------------------------------------------------------------------------------------------------------------- %
    \question{Write a program to compute the GCD of two numbers.}
    \begin{code}
        {Program}{python}
def gcd(a, b):
    # Using the Euclidean algorithm
    if a < b:
        return gcd(a, b - a)
    elif a > b:
        return gcd(a - b, b)
    return a # if a equals b, then gcd is the same as a and b.


print(gcd(
    int(input("Enter first number: ")),
    int(input("Enter second number: "))
))
    \end{code}
    \begin{code}
        {Output}{text}
Enter first number: 10
Enter second number: 4
2
    \end{code}
\end{document}