%! Author = Sujal Singh
%! Date = 11/19/25

% Preamble
%! suppress = FileNotFound
\documentclass[11pt]{ipu-da}
\doctitle{Data Analytics Lab (ARI351)}

% Packages
\usepackage{amsmath}
\usepackage{enumitem}

% Watermark
\usepackage{eso-pic}

% Document
\begin{document}
    \maketitle
    \pagenumbering{gobble}
%\pagestyle{empty}
\begin{center}
    \textbf{\huge Index} \\
    \begin{longtblr}{rows={35pt},colspec={|Q[m,c]|Q[12,m]|Q[2,c]|},hlines,vlines,
        cell{2-Z}{1} = {cmd=\textbf{\the\numexpr\arabic{rownum}-1}.},
        cell{1}{2}={c}}
        \textbf{No.} & \textbf{Question} & \textbf{Remarks} \\
        &%
        \newline
        \noindent\textbf{Write C programs by using Array data structure for the following problem domains:}
        \begin{enumerate}[label=(\alph*)]
            \item Create an array of integer with size $n$.
            Return the difference between the largest and the smallest value inside that array.
            \item Initializes an array with ten random integers and then prints four lines of output, containing: Every
            element at an even index, Every odd element, All elements in reverse order, Only the first and last element.
            \item Consider an integer array of size 5 and display the following: Sum of all the elements, Sum of
            alternate elements in the array, and second-highest element in the array.
        \end{enumerate}
        & \\
        &%
        \newline
        \noindent\textbf{Write a program to create a singly linked list of n nodes and perform:}
        \begin{enumerate}[label=(\alph*)]
            \item Insertion at the beginning.
            \item Insertion at the end.
            \item Insertion at a specific location.
            \item Deletion at the beginning.
            \item Deletion at the end.
            \item Deletion At a specific location.
        \end{enumerate}
        & \\
        &%
        \newline
        \noindent\textbf{Write a program to create a doubly linked list of $n$ nodes and perform:}
        \begin{enumerate}[label=(\alph*)]
            \item Insertion at the beginning.
            \item Insertion at the end.
            \item Insertion at a specific location.
            \item Deletion at the beginning.
            \item Deletion at the end.
            \item Deletion At a specific location.
        \end{enumerate}
        & \\
        &%
        \newline
        \noindent\textbf{Write a program to create a singly circular and doubly linked list of $n$ nodes and perform:}
        \begin{enumerate}[label=(\alph*)]
            \item Insertion at the beginning.
            \item Insertion at the end.
            \item Insertion at a specific location.
            \item Deletion at the beginning.
            \item Deletion at the end.
            \item Deletion At a specific location.
        \end{enumerate}
        & \\
        &%
        Write a program to implement stack using arrays and linked lists.
        & \\
        &%
        Write a program to reverse a sentence/string using stack.
        & \\
        &%
        Write a program to check for balanced parenthesis in a given expression.
        & \\
        &%
        Write a program to convert infix expression to prefix and postfix expression.
        & \\
        &%
        Write a program to implement Linear Queue using Array and Linked Lists.
        & \\
        &%
        Write a program to implement Circular Queue using Array and Linked Lists.
        & \\
        &%
        Write a program to implement Doubly Ended Queue using Array and Linked Lists.
        & \\
        &%
        Write a Program to implement Binary Search Tree operations.
        & \\
        &%
        Write a program to implement Bubble Sort, Selection Sort, Heap Sort, Quick Sort, Merge Sort and Insertion Sort algorithm.
        & \\
        &%
        \newline
        \noindent\textbf{Write C Programs by using Graph data structure for the following problem domains:}
        \begin{enumerate}[label=(\alph*)]
            \item Graph Traversal: BFS
            \item Graph Traversal: DFS
        \end{enumerate}
        & \\
    \end{longtblr}
\end{center}
\newpage%
\pagenumbering{arabic}%


%    % Watermark
%    \AddToShipoutPictureFG{%
%        \begin{tikzpicture}[overlay,remember picture]
%            \path (current page.south west) -- (current page.north east)
%            node[midway,scale=5,color=black,sloped,opacity=0.1] {\textbf{Sujal Singh (04119051723)}};
%        \end{tikzpicture}
%    }

    \experiment{1}{To understand and apply various data pre-processing techniques.}
    \begin{code}{Code}
        {python}
import numpy as np
print("NumPy Version:", np.version )
print("="*50)

# 1. Creating Arrays
print("1. Creating Arrays")
print("-"*30)

# From Python list
arr1 = np.array([1, 2, 3, 4, 5])
print("From list:", arr1)
print("Type:", type(arr1))
print("Data type:", arr1.dtype)

# 2D Array (Matrix)
arr2d = np.array([[1, 2, 3], [4, 5, 6]])
print("\n2D Array:")
print(arr2d)

# Zeros, Ones, Empty
zeros = np.zeros(5)
ones = np.ones((3, 4))
empty = np.empty(4) # uninitialized values
print("\nZeros:", zeros)
print("Ones (3x4) :\n", ones)

# Range of numbers
range_arr = np.arange(0, 20, 2) # start, stop, step
print("arange(0,20,2) :", range_arr)
# Evenly spaced numbers
linspace_arr = np.linspace(0, 1, 5) # 5 numbers from 0 to 1
print("linspace(0,1,5) :", linspace_arr)

# Random numbers
print("\nRandom Arrays:")
print("Random (0-1) :", np.random.random(5))
print("Random integers :", np.random.randint(1, 100, size=6))
print("="*50)

# 2. Array Attributes
print("2. Array Attributes")
print("-"*30)
print("Shape:", arr2d.shape)
print("Size (elements):", arr2d.size)
print("Dimensions:", arr2d.ndim)
print("Item size (bytes):", arr2d.itemsize)
print("Data type:", arr2d.dtype)
print("="*50)

# 3. Basic Operations
print("3. Basic Operations (Element-wise)")
print("-"*30)
a = np.array([10, 20, 30, 40])
b = np.array([1, 2, 3, 4])
print("a:", a)
print("b:", b)
print("a + b:", a + b)
print("a - b:", a - b)
print("a * b:", a * b)
print("a / b:", a / b)
print("a ** 2:", a ** 2)
print("sin(a):", np.sin(a))
print("sqrt(a):", np.sqrt(a))
print("="*50)

# 4. Indexing & Slicing
print("4. Indexing and Slicing")
print("-"*30)
arr = np.array([10, 20, 30, 40, 50, 60])
print("arr:", arr)
print("First element :", arr[0])
print("Last element :", arr[-1])
print("Slice [1:4] :", arr[1:4])
print("Every 2nd:", arr[::2])
print("Reverse:", arr[::-1])

# 2D Indexing
matrix = np.array([[1, 2, 3],
[4, 5, 6],
[7, 8, 9]])
print("\n2D Matrix:\n", matrix)
print("Element [1,2] :", matrix[1, 2]) # row 1, col 2 → 6
print("First row:", matrix[0])
print("First column:", matrix[:, 0])
print("Submatrix 2x2:\n", matrix[:2, :2])
print("="*50)

# 5. Reshaping Arrays
print("5. Reshaping")
print("-"*30)
arr = np.arange(1, 13)
print("Original:", arr)
reshaped = arr.reshape(3, 4) # 3 rows, 4 columns
print("Reshaped 3x4:\n", reshaped)

# Flatten back
flattened = reshaped.flatten()
print("Flattened:", flattened)
print("="*50)

# 6. Mathematical & Statistical Functions
print("6. Math & Stats Functions")
print("-"*30)
data = np.array([12, 15, 18, 22, 30, 35])
print("Data:", data)
print("Mean:", np.mean(data))
print("Median:", np.median(data))
print("Standard Dev:", np.std(data))
print("Variance:", np.var(data))
print("Min:", np.min(data))
print("Max:", np.max(data))
print("Sum:", np.sum(data))
print("Cumulative Sum :", np.cumsum(data))
print("="*50)
    \end{code}
    \begin{code}{Output}
        {text}
NumPy Version: <module 'numpy.version' from '/tmp/.venv/lib64/python3.14/site-packages/numpy/version.py'>
==================================================
1. Creating Arrays
------------------------------
From list: [1 2 3 4 5]
Type: <class 'numpy.ndarray'>
Data type: int64

2D Array:
[[1 2 3]
 [4 5 6]]

Zeros: [0. 0. 0. 0. 0.]
Ones (3x4) :
 [[1. 1. 1. 1.]
 [1. 1. 1. 1.]
 [1. 1. 1. 1.]]
arange(0,20,2) : [ 0  2  4  6  8 10 12 14 16 18]
linspace(0,1,5) : [0.   0.25 0.5  0.75 1.  ]

Random Arrays:
Random (0-1) : [0.50083525 0.09400569 0.44989897 0.74831388 0.57466536]
Random integers : [10 69  1 65 17  9]
==================================================
2. Array Attributes
------------------------------
Shape: (2, 3)
Size (elements): 6
Dimensions: 2
Item size (bytes): 8
Data type: int64
==================================================
3. Basic Operations (Element-wise)
------------------------------
a: [10 20 30 40]
b: [1 2 3 4]
a + b: [11 22 33 44]
a - b: [ 9 18 27 36]
a * b: [ 10  40  90 160]
a / b: [10. 10. 10. 10.]
a ** 2: [ 100  400  900 1600]
sin(a): [-0.54402111  0.91294525 -0.98803162  0.74511316]
sqrt(a): [3.16227766 4.47213595 5.47722558 6.32455532]
==================================================
4. Indexing and Slicing
------------------------------
arr: [10 20 30 40 50 60]
First element : 10
Last element : 60
Slice [1:4] : [20 30 40]
Every 2nd: [10 30 50]
Reverse: [60 50 40 30 20 10]

2D Matrix:
 [[1 2 3]
 [4 5 6]
 [7 8 9]]
Element [1,2] : 6
First row: [1 2 3]
First column: [1 4 7]
Submatrix 2x2:
 [[1 2]
 [4 5]]
==================================================
5. Reshaping
------------------------------
Original: [ 1  2  3  4  5  6  7  8  9 10 11 12]
Reshaped 3x4:
 [[ 1  2  3  4]
 [ 5  6  7  8]
 [ 9 10 11 12]]
Flattened: [ 1  2  3  4  5  6  7  8  9 10 11 12]
==================================================
6. Math & Stats Functions
------------------------------
Data: [12 15 18 22 30 35]
Mean: 22.0
Median: 20.0
Standard Dev: 8.144527815247077
Variance: 66.33333333333333
Min: 12
Max: 35
Sum: 132
Cumulative Sum : [ 12  27  45  67  97 132]
==================================================
    \end{code}

    \experiment{2}{To understand basic matplotlib for plotting bar, line, pie, histogram etc.}
    \begin{code}{Code}
        {python}
import matplotlib.pyplot as plt
import numpy as np

plt.style.use('seaborn-v0_8')

months = ['Jan', 'Feb', 'Mar', 'Apr', 'May', 'Jun']
sales = [120, 135, 180, 165, 200, 220]
profit = [30, 40, 55, 50, 70, 80]
categories = ['Rent', 'Food', 'Transport', 'Entertainment', 'Savings']
expenses = [5000, 3000, 1500, 2000, 4000]
ages = np.random.normal(35, 10, 1000)

# Line Plot plt.figure(figsize=(10,6))
plt.plot(months, sales, 'o-b', label='Sales')
plt.plot(months, profit, 's-g', label='Profit')
plt.title('Monthly Sales vs Profit')
plt.xlabel('Month')
plt.ylabel('Amount (₹)')
plt.legend()
plt.grid(True)
plt.show()

# Bar Plot plt.figure(figsize=(10,6))
x = np.arange(len(months))
plt.bar(x-0.2, sales, 0.4, label='Sales', color='skyblue')
plt.bar(x+0.2, profit, 0.4, label='Profit', color='lightgreen')
plt.title('Sales & Profit by Month')
plt.xlabel('Month')
plt.ylabel('Amount (₹)')
plt.xticks(x, months)
plt.legend()
plt.show()

# Horizontal Bar plt.figure(figsize=(10,6))
plt.barh(categories, expenses, color='coral')
plt.title('Monthly Expenses')
plt.xlabel('Amount (₹)')
plt.show()

# Pie Chart plt.figure(figsize=(8,8))
plt.pie(expenses, labels=categories, autopct='%1.1f%%', startangle=90)
plt.title('Expense Distribution')
plt.axis('equal')
plt.show()

# Histogram plt.figure(figsize=(10,6))
plt.hist(ages, bins=30, color='purple', edgecolor='black', alpha=0.7)
plt.title('Age Distribution')
plt.xlabel('Age')
plt.ylabel('Count')
plt.show()
# Scatter Plot plt.figure(figsize=(10,6))
plt.scatter(np.random.rand(100), np.random.rand(100), c=np.random.rand(100), s=200, alpha=0.6, cmap='viridis')
plt.title('Scatter Plot')
plt.show()

# Subplots
fig, axes = plt.subplots(2, 2, figsize=(12,10))
axes[0,0].plot(months, sales, 'o-r')
axes[0,0].set_title('Line Plot')

axes[0,1].bar(categories, expenses, color='orange')
axes[0,1].set_title('Bar Plot')

axes[1,0].pie(expenses, labels=categories, autopct='%1.0f%%')
axes[1,0].set_title('Pie Chart')
axes[1,1].hist(ages, bins=25, color='teal')
axes[1,1].set_title('Histogram')

plt.tight_layout()
plt.show()
    \end{code}
    \begin{tabularsection}{Plots}%
        \includegraphics[width=\textwidth]{figures/2_1}
        \includegraphics[width=\textwidth]{figures/2_2}
        \includegraphics[width=\textwidth]{figures/2_3}
        \includegraphics[width=\textwidth]{figures/2_4}
        \includegraphics[width=\textwidth]{figures/2_5}
        \includegraphics[width=\textwidth]{figures/2_6}
        \includegraphics[width=\textwidth]{figures/2_7}
    \end{tabularsection}%

    \experiment{3}{Understanding data preprocessing using Pandas.}
    \begin{code}{Code}
        {python}
import pandas as pd
import seaborn as sns
import matplotlib.pyplot as plt

df = pd.read_csv("./iris.csv")
print("Shape (rows, columns) :", df.shape)
print("Total samples:", df.shape[0])
print("Total features:", df.shape[1])
print("\nColumns:")
print(df.columns.tolist())

print("\nFirst 5 rows:")
print(df.head())

print("\nData types & missing values:")
print(df.info())

print("\nMissing values per column:")
print(df.isnull().sum())

target_col = df.columns[-1] # assumes last column is target
print(f"\nTarget column: '{target_col}'")
print("Number of classes :", df[target_col].nunique())
print("\nClass distribution:")
print(df[target_col].value_counts())

# 1. Countplot
plt.figure(figsize=(10,5))
sns.countplot(data=df, x=target_col, palette="viridis")
plt.title(f"Countplot of {target_col}")
plt.xticks(rotation=45)
plt.show()

# 2. Pairplot (only if dataset is small)
if df.shape[1] <= 12:
    sns.pairplot(df, hue=target_col, diag_kind='kde', corner=True)
    plt.show()

# 3. Correlation Heatmap (numeric columns only)
numeric_df = df.select_dtypes(include=['float64', 'int64'])
plt.figure(figsize=(10,8))
sns.heatmap(numeric_df.corr(), annot=True, cmap='coolwarm', center=0, linewidths=0.5)
plt.title("Correlation Heatmap")
plt.show()

# 4. Statistical Summary
print("\nStatistical Summary (Numeric):")
print(df.describe())
print("\nAll done! Data fully explored.")
    \end{code}
    \begin{code}{Output}
        {text}
Shape (rows, columns) : (150, 5)
Total samples: 150
Total features: 5

Columns:
['sepal.length', 'sepal.width', 'petal.length', 'petal.width', 'variety']

First 5 rows:
   sepal.length  sepal.width  petal.length  petal.width variety
0           5.1          3.5           1.4          0.2  Setosa
1           4.9          3.0           1.4          0.2  Setosa
2           4.7          3.2           1.3          0.2  Setosa
3           4.6          3.1           1.5          0.2  Setosa
4           5.0          3.6           1.4          0.2  Setosa

Data types & missing values:
<class 'pandas.core.frame.DataFrame'>
RangeIndex: 150 entries, 0 to 149
Data columns (total 5 columns):
 #   Column        Non-Null Count  Dtype
---  ------        --------------  -----
 0   sepal.length  150 non-null    float64
 1   sepal.width   150 non-null    float64
 2   petal.length  150 non-null    float64
 3   petal.width   150 non-null    float64
 4   variety       150 non-null    object
dtypes: float64(4), object(1)
memory usage: 6.0+ KB
None

Missing values per column:
sepal.length    0
sepal.width     0
petal.length    0
petal.width     0
variety         0
dtype: int64

Target column: 'variety'
Number of classes : 3

Class distribution:
variety
Setosa        50
Versicolor    50
Virginica     50
Name: count, dtype: int64
    \end{code}
    \begin{tabularsection}{Plots}%
        \includegraphics[width=\textwidth]{figures/3_1}
    \end{tabularsection}%
    \vfill
    \begin{tabularsection}{Plots}%
        \includegraphics[width=\textwidth]{figures/3_2}
    \end{tabularsection}%

    \experiment{4 (a)}{Understand and apply regression algorithm.}
    \begin{code}{Code}
        {python}
import pandas as pd
import matplotlib.pyplot as plt
import seaborn as sns
from sklearn.datasets import load_iris
from sklearn.model_selection import train_test_split
from sklearn.linear_model import LinearRegression
from sklearn.metrics import mean_squared_error, r2_score, mean_absolute_error

# Load Iris dataset correctly
iris = load_iris(as_frame=True)
df = iris.frame

# Check actual column names
print("Columns in dataset:", df.columns.tolist())
print("\nFirst 5 rows:")
print(df.head())

# Correct column names
X = df[['petal length (cm)']]
y = df['petal width (cm)']

# Split data
X_train, X_test, y_train, y_test = train_test_split(X, y, test_size=0.2, random_state=42)

# Train model
model = LinearRegression()
model.fit(X_train, y_train)

# Predict
y_pred = model.predict(X_test)

# Results
print("\n" + "="*60)
print("LINEAR REGRESSION: petal length → petal width")
print("="*60)
print(f"Slope: {model.coef_[0]:.4f}")
print(f"Intercept: {model.intercept_:.4f}")
print(f"R² Score: {r2_score(y_test, y_pred):.4f}")

# Plot
plt.figure(figsize=(10,6))
plt.scatter(X, y, color='teal', alpha=0.7, label='Actual data')
plt.plot(
    X_test.sort_values(by='petal length (cm)'),
    model.predict(X_test.sort_values(by='petal length (cm)')),
    color='red',
    linewidth=3,
    label='Regression Line'
)
plt.xlabel('Petal Length (cm)')
plt.ylabel('Petal Width (cm)')
plt.title('Iris Dataset - Linear Regression\n(petal length → petal width)')
plt.legend()
plt.grid(True, alpha=0.3)
plt.show()

# Predict new values
new = [[1.4], [4.0], [5.5]]
for length in new:
    pred = model.predict([length])
    print(f"Petal length {length[0]} cm → Predicted width: {pred[0]:.3f} cm")
    \end{code}
    \begin{code}{Output}
        {text}
Columns in dataset: ['sepal length (cm)', 'sepal width (cm)', 'petal length (cm)', 'petal width (cm)', 'target']

First 5 rows:
   sepal length (cm)  sepal width (cm)  ...  petal width (cm)  target
0                5.1               3.5  ...               0.2       0
1                4.9               3.0  ...               0.2       0
2                4.7               3.2  ...               0.2       0
3                4.6               3.1  ...               0.2       0
4                5.0               3.6  ...               0.2       0

[5 rows x 5 columns]

============================================================
LINEAR REGRESSION: petal length → petal width
============================================================
Slope: 0.4132
Intercept: -0.3567
R² Score: 0.9283
Petal length 1.4 cm → Predicted width: 0.222 cm
Petal length 4.0 cm → Predicted width: 1.296 cm
Petal length 5.5 cm → Predicted width: 1.916 cm
    \end{code}
    \vfill
    \begin{tabularsection}{Plots}
        \includegraphics[width=\textwidth]{figures/4_a}
    \end{tabularsection}

    \experiment{4 (b)}{Apply regression algorithm and handle missing value and categorical data.}
    \begin{code}{Code}
        {python}
import pandas as pd
import numpy as np
import matplotlib.pyplot as plt
import seaborn as sns
from sklearn.datasets import load_iris
from sklearn.model_selection import train_test_split
from sklearn.pipeline import Pipeline
from sklearn.compose import ColumnTransformer
from sklearn.impute import SimpleImputer
from sklearn.preprocessing import StandardScaler, OneHotEncoder
from sklearn.linear_model import LinearRegression
from sklearn.metrics import r2_score, mean_absolute_error

# Load Iris
iris = load_iris(as_frame=True)
df = iris.frame

# Add species name as categorical column
df['species'] = pd.Categorical.from_codes(iris.target, iris.target_names)

# Intentionally create missing values (for learning)
np.random.seed(42)
df.loc[np.random.choice(df.index, 20), 'sepal length (cm)'] = np.nan
df.loc[np.random.choice(df.index, 15), 'petal length (cm)'] = np.nan
print("Missing values created:")
print(df.isnull().sum())

# Target: predict petal_width
X = df.drop(['petal width (cm)', 'target'], axis=1)
y = df['petal width (cm)']

# Split
X_train, X_test, y_train, y_test = train_test_split(X, y, test_size=0.2, random_state=42)

# Preprocessing
numeric_features = ['sepal length (cm)', 'sepal width (cm)', 'petal length (cm)']
categorical_features = ['species']
numeric_pipeline = Pipeline([
    ('imputer', SimpleImputer(strategy='mean')),
    ('scaler', StandardScaler())
])
categorical_pipeline = Pipeline([
    ('imputer', SimpleImputer(strategy='most_frequent')),
    ('onehot', OneHotEncoder(drop='first'))
])
preprocessor = ColumnTransformer([
    ('num', numeric_pipeline, numeric_features),
    ('cat', categorical_pipeline, categorical_features)
])

# Full model
model = Pipeline([
    ('preprocessor', preprocessor),
    ('regressor', LinearRegression())
])

# Train
model.fit(X_train,y_train)

# Predict
y_pred = model.predict(X_test)

# Results
print("\n" + "="*50)
print("REGRESSION ON IRIS (with missing + categorical)")
print("="*50)
print(f"R² Score: {r2_score(y_test, y_pred):.4f}")
print(f"MAE: {mean_absolute_error(y_test, y_pred):.4f}")

# Plot
plt.figure(figsize=(8,6))
plt.scatter(y_test, y_pred, color='purple', alpha=0.7)
plt.plot([y.min(), y.max()], [y.min(), y.max()], 'r--')
plt.xlabel('Actual Petal Width (cm)')
plt.ylabel('Predicted Petal Width (cm)')
plt.title('Linear Regression on Iris\n(Missing Values + Species Included)')
plt.grid(True, alpha=0.3)
plt.show()

# Predict on new flower
new_flower = pd.DataFrame({
    'sepal length (cm)': [5.9],
    'sepal width (cm)': [3.0],
    'petal length (cm)': [np.nan],
    'species':
     ['virginica']
})

# missing
pred = model.predict(new_flower)[0]
print(f"\nPredicted petal width = {pred:.3f} cm")

sns.pairplot(
    df[['sepal length (cm)', 'sepal width (cm)', 'petal length (cm)', 'petal width (cm)', 'species']],
    diag_kind='kde'
)
plt.show()
    \end{code}
    \begin{code}{Output}
        {text}
Missing values created:
sepal length (cm)    18
sepal width (cm)      0
petal length (cm)    13
petal width (cm)      0
target                0
species               0
dtype: int64

==================================================
REGRESSION ON IRIS (with missing + categorical)
==================================================
R² Score: 0.9608
MAE: 0.1278

Predicted petal width = 1.855 cm
    \end{code}
    \vfill
    \begin{tabularsection}{Plot}
        \includegraphics[width=\textwidth]{figures/4_b_1}
    \end{tabularsection}
    \begin{tabularsection}{Plot}
        \includegraphics[width=\textwidth]{figures/4_b_2}
    \end{tabularsection}

    \experiment{5}{Program to implement feature scaling, feature extraction and selection.}
    \begin{code}{Code}
        {python}
import pandas as pd
import numpy as np
import matplotlib.pyplot as plt
import seaborn as sns
from sklearn.datasets import load_iris
from sklearn.model_selection import train_test_split
from sklearn.preprocessing import StandardScaler, MinMaxScaler
from sklearn.decomposition import PCA
from sklearn.feature_selection import SelectKBest, f_classif, RFE
from sklearn.linear_model import LogisticRegression
from sklearn.metrics import accuracy_score

# Load Iris
iris = load_iris(as_frame=True)
X = iris.data
y = iris.target
feature_names = iris.feature_names
print("Original Features:", feature_names)
print("Shape:", X.shape)
X_train, X_test, y_train, y_test = train_test_split(X, y, test_size=0.3, random_state=42)

# 1. No Scaling (Baseline)
model = LogisticRegression(max_iter=200)
model.fit(X_train, y_train)
acc0 = accuracy_score(y_test, model.predict(X_test))
print(f"\n1. No Scaling → Accuracy: {acc0:.4f}")

# 2. Standard Scaling (Zero mean, unit variance)
scaler_std = StandardScaler()
X_train_std = scaler_std.fit_transform(X_train)
X_test_std = scaler_std.transform(X_test)
model.fit(X_train_std, y_train)
acc1 = accuracy_score(y_test, model.predict(X_test_std))
print(f"2. StandardScaler → Accuracy: {acc1:.4f}")

# 3. MinMax Scaling (0 to 1)
scaler_mm = MinMaxScaler()
X_train_mm = scaler_mm.fit_transform(X_train)
X_test_mm = scaler_mm.transform(X_test)
model.fit(X_train_mm, y_train)
acc2 = accuracy_score(y_test, model.predict(X_test_mm))
print(f"3. MinMaxScaler → Accuracy: {acc2:.4f}")

# 4. Feature Extraction using PCA (reduce to 2 components)
pca = PCA(n_components=2)
X_train_pca = pca.fit_transform(X_train_std)
X_test_pca = pca.transform(X_test_std)

model.fit(X_train_pca, y_train)
acc3 = accuracy_score(y_test, model.predict(X_test_pca))
print(f"4. PCA (2 components) → Accuracy: {acc3:.4f}")
print(f" Explained Variance: {pca.explained_variance_ratio_.sum():.4f}")

# 5. Feature Selection - SelectKBest (top 2 features)
selector_k = SelectKBest(score_func=f_classif, k=2)
X_train_k = selector_k.fit_transform(X_train_std, y_train)
X_test_k = selector_k.transform(X_test_std)

model.fit(X_train_k, y_train)
acc4 = accuracy_score(y_test, model.predict(X_test_k))
print(f"5. SelectKBest (k=2) → Accuracy: {acc4:.4f}")
print(" Selected features:", [feature_names[i] for i in selector_k.get_support(indices=True)])

# 6. Recursive Feature Elimination (RFE)
rfe = RFE(estimator=LogisticRegression(max_iter=200), n_features_to_select=2)
X_train_rfe = rfe.fit_transform(X_train_std, y_train)
X_test_rfe = rfe.transform(X_test_std)

model.fit(X_train_rfe, y_train)
acc5 = accuracy_score(y_test, model.predict(X_test_rfe))
print(f"6. RFE (2 features) → Accuracy: {acc5:.4f}")
print(" Selected features:", [feature_names[i] for i in range(len(rfe.support_)) if rfe.support_[i]])

# Final Comparison Plot
results = pd.DataFrame({
'Method': ['No Scaling', 'StandardScaler', 'MinMaxScaler', 'PCA', 'SelectKBest', 'RFE'],
'Accuracy': [acc0, acc1, acc2, acc3, acc4, acc5]
})

plt.figure(figsize=(10,6))
sns.barplot(x='Accuracy', y='Method', data=results, palette='viridis')
plt.title('Feature Scaling & Selection Comparison on Iris')
plt.xlim(0.8, 1.0)

for i, v in enumerate(results['Accuracy']):
    plt.text(v + 0.001, i, f"{v:.4f}", va='center')
plt.show()
    \end{code}
    \begin{code}{Output}
        {text}
Original Features: ['sepal length (cm)', 'sepal width (cm)', 'petal length (cm)', 'petal width (cm)']
Shape: (150, 4)

1. No Scaling → Accuracy: 1.0000
2. StandardScaler → Accuracy: 1.0000
3. MinMaxScaler → Accuracy: 0.9111
4. PCA (2 components) → Accuracy: 0.9111
 Explained Variance: 0.9521
5. SelectKBest (k=2) → Accuracy: 1.0000
 Selected features: ['petal length (cm)', 'petal width (cm)']
6. RFE (2 features) → Accuracy: 1.0000
 Selected features: ['petal length (cm)', 'petal width (cm)']
    \end{code}
    \begin{tabularsection}{Plots}
        \includegraphics[width=\textwidth]{figures/5}
    \end{tabularsection}


    \experiment{6}{Program to implement simple and multiple linear regression.}
    \begin{code}{Code}
        {python}
import pandas as pd
import numpy as np
import matplotlib.pyplot as plt
import seaborn as sns
from sklearn.datasets import load_iris
from sklearn.model_selection import train_test_split
from sklearn.linear_model import LinearRegression
from sklearn.metrics import r2_score, mean_absolute_error, mean_squared_error

# Load Iris
iris = load_iris(as_frame=True)
df = iris.frame
df.columns = ['sepal_length', 'sepal_width', 'petal_length', 'petal_width', 'species']

print("Dataset Shape:", df.shape)
print(df.head())

# Target: petal_width
y = df['petal_width']

# 1. SIMPLE LINEAR REGRESSION (Best single feature: petal_length)
X_simple = df[['petal_length']]

X_train_s, X_test_s, y_train_s, y_test_s = train_test_split(X_simple, y, test_size=0.2, random_state=42)

model_simple = LinearRegression()
model_simple.fit(X_train_s, y_train_s)
y_pred_s = model_simple.predict(X_test_s)

print("\n" + "="*60)
print("1. SIMPLE LINEAR REGRESSION (petal_length → petal_width)")
print("="*60)
print(f"Slope: {model_simple.coef_[0]:.4f}")
print(f"Intercept: {model_simple.intercept_:.4f}")
print(f"R² Score: {r2_score(y_test_s, y_pred_s):.4f}")
print(f"RMSE: {np.sqrt(mean_squared_error(y_test_s, y_pred_s)):.4f}")

# Plot Simple Regression
plt.figure(figsize=(10,5))
plt.scatter(X_test_s, y_test_s, color='blue', label='Actual')
plt.plot(X_test_s, y_pred_s, color='red', linewidth=3, label='Prediction')
plt.title('Simple Linear Regression\n(petal_length → petal_width)')
plt.xlabel('Petal Length (cm)')
plt.ylabel('Petal Width (cm)')
plt.legend()
plt.grid(True, alpha=0.3)
plt.show()

# 2. MULTIPLE LINEAR REGRESSION (All 3 numeric features)
X_multiple = df[['sepal_length', 'sepal_width', 'petal_length']]

X_train_m, X_test_m, y_train_m, y_test_m = train_test_split(X_multiple, y, test_size=0.2, random_state=42)

model_multiple = LinearRegression()
model_multiple.fit(X_train_m, y_train_m)
y_pred_m = model_multiple.predict(X_test_m)

print("\n" + "="*60)
print("2. MULTIPLE LINEAR REGRESSION (3 features → petal_width)")
print("="*60)
print(f"Coefficients : {model_multiple.coef_.round(4)}")
print(f"Intercept: {model_multiple.intercept_:.4f}")
print(f"R² Score: {r2_score(y_test_m, y_pred_m):.4f}")

# Feature importance (coefficient magnitude)
features = X_multiple.columns
coefs = np.abs(model_multiple.coef_)
plt.figure(figsize=(8,5))
sns.barplot(x=coefs, y=features, palette='magma')
plt.title('Feature Importance in Multiple Regression')
plt.xlabel('Absolute Coefficient Value')
plt.show()

# Actual vs Predicted (Multiple)
plt.figure(figsize=(8,6))
plt.scatter(y_test_m, y_pred_m, color='green', alpha=0.7)
plt.plot([y.min(), y.max()], [y.min(), y.max()], 'r--')
plt.xlabel('Actual Petal Width')
plt.ylabel('Predicted Petal Width')
plt.title('Multiple Linear Regression: Actual vs Predicted')
plt.grid(True, alpha=0.3)
plt.show()

# FINAL COMPARISON
print("\n" + "="*60)
print("COMPARISON")
print("="*60)
print(f"Simple Linear Regression R² = {r2_score(y_test_s, y_pred_s):.4f}")
print(f"Multiple Linear Regression R² = {r2_score(y_test_m, y_pred_m):.4f}")
    \end{code}
    \begin{code}{Output}
        {text}
Dataset Shape: (150, 5)
   sepal_length  sepal_width  petal_length  petal_width  species
0           5.1          3.5           1.4          0.2        0
1           4.9          3.0           1.4          0.2        0
2           4.7          3.2           1.3          0.2        0
3           4.6          3.1           1.5          0.2        0
4           5.0          3.6           1.4          0.2        0

============================================================
1. SIMPLE LINEAR REGRESSION (petal_length → petal_width)
============================================================
Slope: 0.4132
Intercept: -0.3567
R² Score: 0.9283
RMSE: 0.2136

============================================================
2. MULTIPLE LINEAR REGRESSION (3 features → petal_width)
============================================================
Coefficients : [-0.2343  0.2359  0.5343]
Intercept: -0.1693
R² Score: 0.9271

============================================================
COMPARISON
============================================================
Simple Linear Regression R² = 0.9283
Multiple Linear Regression R² = 0.9271
    \end{code}
    \begin{tabularsection}{Plots}
        \includegraphics[width=\textwidth]{figures/6_1}
        \includegraphics[width=\textwidth]{figures/6_2}
        \includegraphics[width=\textwidth]{figures/6_3}
    \end{tabularsection}

    \experiment{7}{Program to implement regularized and non linear regression.}
    \begin{code}{Code}
        {python}
import pandas as pd
import numpy as np
import matplotlib.pyplot as plt
import seaborn as sns
from sklearn.datasets import load_iris
from sklearn.model_selection import train_test_split
from sklearn.linear_model import LinearRegression, Ridge, Lasso
from sklearn.preprocessing import PolynomialFeatures
from sklearn.pipeline import make_pipeline
from sklearn.metrics import r2_score, mean_squared_error

# Load Iris
iris = load_iris(as_frame=True)
df = iris.frame
df.columns = ['sepal_length', 'sepal_width', 'petal_length', 'petal_width', 'species']

X = df[['petal_length']] # Feature
y = df['petal_width'] # Target

X_train, X_test, y_train, y_test = train_test_split(X, y, test_size=0.2, random_state=42)

results = []

# 1. Ordinary Linear Regression
lr = LinearRegression()
lr.fit(X_train, y_train)
y_pred_lr = lr.predict(X_test)
r2_lr = r2_score(y_test, y_pred_lr)
results.append(['Linear', r2_lr])

# 2. Ridge Regression (L2)
ridge = Ridge(alpha=1.0)
ridge.fit(X_train, y_train)
y_pred_ridge = ridge.predict(X_test)
r2_ridge = r2_score(y_test, y_pred_ridge)
results.append(['Ridge (L2)', r2_ridge])

# 3. Lasso Regression (L1)
lasso = Lasso(alpha=0.01, max_iter=10000)
lasso.fit(X_train, y_train)
y_pred_lasso = lasso.predict(X_test)
r2_lasso = r2_score(y_test, y_pred_lasso)
results.append(['Lasso (L1)', r2_lasso])

# 4. Polynomial Regression (Degree 2 & 3)
poly2 = make_pipeline(PolynomialFeatures(degree=2), LinearRegression())
poly2.fit(X_train, y_train)
y_pred_p2 = poly2.predict(X_test)
r2_p2 = r2_score(y_test, y_pred_p2)
results.append(['Polynomial (deg=2)', r2_p2])

poly3 = make_pipeline(PolynomialFeatures(degree=3), LinearRegression())
poly3.fit(X_train, y_train)
y_pred_p3 = poly3.predict(X_test)
r2_p3 = r2_score(y_test, y_pred_p3)
results.append(['Polynomial (deg=3)', r2_p3])

# Print Results
print("="*65)
print("REGRESSION COMPARISON ON IRIS (petal_length → petal_width)")
print("="*65)
for name, score in results:
    print(f"{name:25} → R² = {score:.5f}")
print("="*65)

# Plot all models
plt.figure(figsize=(12,8))
plt.scatter(X, y, color='blue', alpha=0.6, label='Actual Data')

# Sort for smooth lines
X_plot = np.linspace(X.min(), X.max(), 100).reshape(-1,1)

plt.plot(X_plot, lr.predict(X_plot), 'g-', linewidth=3, label=f'Linear (R²={r2_lr:.4f})')
plt.plot(X_plot, ridge.predict(X_plot), 'r--', linewidth=2, label=f'Ridge (R²={r2_ridge:.4f})')
plt.plot(X_plot, lasso.predict(X_plot), 'm:', linewidth=2, label=f'Lasso (R²={r2_lasso:.4f})')
plt.plot(X_plot, poly2.predict(X_plot), 'c-', linewidth=2, label=f'Poly deg2 (R²={r2_p2:.4f})')
plt.plot(X_plot, poly3.predict(X_plot), 'orange', linewidth=2, label=f'Poly deg3 (R²={r2_p3:.4f})')

plt.xlabel('Petal Length (cm)')
plt.ylabel('Petal Width (cm)')
plt.title('Regularized + Non-Linear Regression Comparison')
plt.legend()
plt.grid(True, alpha=0.3)
plt.show()

# Coefficient comparison
print(f"\nCoefficients:")
print(f"Linear : {lr.coef_[0]:.4f}")
print(f"Ridge : {ridge.coef_[0]:.4f}")
print(f"Lasso : {lasso.coef_[0]:.4f}")
    \end{code}
    \begin{code}{Output}
        {text}
=================================================================
REGRESSION COMPARISON ON IRIS (petal_length → petal_width)
=================================================================
Linear                    → R² = 0.92826
Ridge (L2)                → R² = 0.92811
Lasso (L1)                → R² = 0.92779
Polynomial (deg=2)        → R² = 0.92834
Polynomial (deg=3)        → R² = 0.93758
=================================================================

Coefficients:
Linear : 0.4132
Ridge : 0.4121
Lasso : 0.4100
    \end{code}
    \begin{tabularsection}{Plots}
        \includegraphics[width=\textwidth]{figures/7}
    \end{tabularsection}

    \experiment{8}{Program to implement K-Means clustering techniques.}
    \begin{code}{Code}
        {python}
import matplotlib.pyplot as plt
import numpy as np
from scipy.cluster.hierarchy import dendrogram,linkage
from sklearn.datasets import make_blobs
from sklearn.cluster import AgglomerativeClustering
from sklearn.cluster import DBSCAN
from sklearn.cluster import KMeans
from sklearn.datasets import make_moons


x,y=make_blobs(n_samples=150,n_features=2,centers=3,cluster_std=0.5,

shuffle=True,random_state=0)

plt.scatter(x[:,0],x[:,1],c='white',edgecolor='black',marker='o',s=50)
plt.show()

km=KMeans(n_clusters=3,init='random',max_iter=100,n_init=10)
y_km=km.fit_predict(x)
print('Distortion=',km.inertia_)
print(km.cluster_centers_)

plt.figure()
plt.scatter(
    x[y_km==0,0], x[y_km==0,1], s=50,
    c='lightgreen', marker='d', edgecolor='black',
    label='cluster-1'
)
plt.scatter(
    x[y_km==1,0], x[y_km==1,1], s=50,
    c='orange', marker='o', edgecolor='black',
    label='cluster-2'
)
plt.scatter(
    x[y_km==2,0], x[y_km==2,1], s=50,
    c='blue', marker='>', edgecolor='black',
    label='cluster-3'
)
plt.scatter(
    km.cluster_centers_[:,0], km.cluster_centers_[:,1], s=100,
    c='red', marker='*', edgecolor='black',
    label='centroids'
)
plt.legend()
plt.show()

dist=[]
for i in range(1,11):
    km=KMeans(n_clusters=i,init='k-means++',n_init=10,max_iter=100)
    km.fit(x)
    dist.append(km.inertia_)
plt.plot(range(1,11),dist,marker='o')
plt.xlabel('Number of clusters (K)-->')
plt.ylabel('Distortions')
plt.show()

# Agglomerative
x=np.array([[1,2],[1,4],[1,0],[4,2],[4,4],[4,0]])
plt.scatter(x[:,0],x[:,1])
plt.show()
ac=AgglomerativeClustering(n_clusters=2,metric='euclidean',linkage='single') # 'complete'
labels=ac.fit_predict(x)
print(labels)

x=np.array([[1,2],[1,4],[1,0],[4,2],[4,4],[4,0]])
z=linkage(x,'single') # 'complete'
dendrogram(z)
plt.title('Hierarchical clustering')
plt.xlabel('Data Points')
plt.ylabel('Distance')
plt.show()

x,y=make_moons(n_samples=200,noise=0.05,random_state=0)
plt.scatter(x[:,0],x[:,1])
plt.show()

km=KMeans(n_clusters=2,random_state=0)
y_km=km.fit_predict(x)
plt.scatter(x[y_km==0,0],x[y_km==0,1],c='blue',s=50,marker='o')
plt.scatter(x[y_km==1,0],x[y_km==1,1],c='red',s=50,marker='d')
plt.scatter(
    km.cluster_centers_[:,0],
    km.cluster_centers_[:,1],
    c='black',s=50,marker='*'
)
plt.show()

km=AgglomerativeClustering(n_clusters=2,linkage='single') #'compete
y_km=km.fit_predict(x)
plt.scatter(x[y_km==0,0],x[y_km==0,1],c='blue',s=50,marker='o')
plt.scatter(x[y_km==1,0],x[y_km==1,1],c='red',s=50,marker='d')
plt.show()

km=DBSCAN(eps=0.2,min_samples=5,metric='euclidean')
y_km=km.fit_predict(x)
plt.scatter(x[y_km==0,0],x[y_km==0,1],c='blue',s=50,marker='o')
plt.scatter(x[y_km==1,0],x[y_km==1,1],c='red',s=50,marker='d')
plt.show()
    \end{code}
    \begin{code}{Output}
        {text}
Distortion= 72.47601670996698
[[-1.5947298   2.92236966]
 [ 2.06521743  0.96137409]
 [ 0.9329651   4.35420712]]
[1 1 1 0 0 0]
    \end{code}
    \newpage
    \begin{tabularsection}{Plots}
        \begin{center}
            \includegraphics[width=0.8\textwidth]{figures/8_1}
            \includegraphics[width=0.8\textwidth]{figures/8_2}
            \includegraphics[width=0.8\textwidth]{figures/8_3}
            \includegraphics[width=0.8\textwidth]{figures/8_4}
            \includegraphics[width=0.8\textwidth]{figures/8_5}
            \includegraphics[width=0.8\textwidth]{figures/8_6}
            \includegraphics[width=0.8\textwidth]{figures/8_7}
            \includegraphics[width=0.8\textwidth]{figures/8_8}
            \includegraphics[width=0.8\textwidth]{figures/8_9}
        \end{center}
    \end{tabularsection}
\end{document}