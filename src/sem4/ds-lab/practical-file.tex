%! Author = Sujal Singh
%! Date = 5/6/25

% Preamble
%! suppress = FileNotFound
\documentclass[11pt]{ipu-ai}
\doctitle{Data Structures Lab (ARI254)}

% Packages
\usepackage{amsmath}
\usepackage{enumitem}


\renewcommand{\experiment}[3]{%
    \newpage%
    \begin{center}%
    \textbf{\Huge Experiment--#1}\\[30pt]%
    \end{center}%
    \begin{tabularsection}{Aim}%
    ~\\#2\\%
    \end{tabularsection}}%

% Document
\begin{document}
    \maketitle
    \pagenumbering{gobble}
%\pagestyle{empty}
\begin{center}
    \textbf{\huge Index} \\
    \begin{longtblr}{rows={35pt},colspec={|Q[m,c]|Q[12,m]|Q[2,c]|},hlines,vlines,
        cell{2-Z}{1} = {cmd=\textbf{\the\numexpr\arabic{rownum}-1}.},
        cell{1}{2}={c}}
        \textbf{No.} & \textbf{Question} & \textbf{Remarks} \\
        &%
        \newline
        \noindent\textbf{Write C programs by using Array data structure for the following problem domains:}
        \begin{enumerate}[label=(\alph*)]
            \item Create an array of integer with size $n$.
            Return the difference between the largest and the smallest value inside that array.
            \item Initializes an array with ten random integers and then prints four lines of output, containing: Every
            element at an even index, Every odd element, All elements in reverse order, Only the first and last element.
            \item Consider an integer array of size 5 and display the following: Sum of all the elements, Sum of
            alternate elements in the array, and second-highest element in the array.
        \end{enumerate}
        & \\
        &%
        \newline
        \noindent\textbf{Write a program to create a singly linked list of n nodes and perform:}
        \begin{enumerate}[label=(\alph*)]
            \item Insertion at the beginning.
            \item Insertion at the end.
            \item Insertion at a specific location.
            \item Deletion at the beginning.
            \item Deletion at the end.
            \item Deletion At a specific location.
        \end{enumerate}
        & \\
        &%
        \newline
        \noindent\textbf{Write a program to create a doubly linked list of $n$ nodes and perform:}
        \begin{enumerate}[label=(\alph*)]
            \item Insertion at the beginning.
            \item Insertion at the end.
            \item Insertion at a specific location.
            \item Deletion at the beginning.
            \item Deletion at the end.
            \item Deletion At a specific location.
        \end{enumerate}
        & \\
        &%
        \newline
        \noindent\textbf{Write a program to create a singly circular and doubly linked list of $n$ nodes and perform:}
        \begin{enumerate}[label=(\alph*)]
            \item Insertion at the beginning.
            \item Insertion at the end.
            \item Insertion at a specific location.
            \item Deletion at the beginning.
            \item Deletion at the end.
            \item Deletion At a specific location.
        \end{enumerate}
        & \\
        &%
        Write a program to implement stack using arrays and linked lists.
        & \\
        &%
        Write a program to reverse a sentence/string using stack.
        & \\
        &%
        Write a program to check for balanced parenthesis in a given expression.
        & \\
        &%
        Write a program to convert infix expression to prefix and postfix expression.
        & \\
        &%
        Write a program to implement Linear Queue using Array and Linked Lists.
        & \\
        &%
        Write a program to implement Circular Queue using Array and Linked Lists.
        & \\
        &%
        Write a program to implement Doubly Ended Queue using Array and Linked Lists.
        & \\
        &%
        Write a Program to implement Binary Search Tree operations.
        & \\
        &%
        Write a program to implement Bubble Sort, Selection Sort, Heap Sort, Quick Sort, Merge Sort and Insertion Sort algorithm.
        & \\
        &%
        \newline
        \noindent\textbf{Write C Programs by using Graph data structure for the following problem domains:}
        \begin{enumerate}[label=(\alph*)]
            \item Graph Traversal: BFS
            \item Graph Traversal: DFS
        \end{enumerate}
        & \\
    \end{longtblr}
\end{center}
\newpage%
\pagenumbering{arabic}%

    % ---------------------------------------------------------------------------------------------------------------- %

    %------------------------------------------------------------------------------------------------------------------%
    %------------------------------------------------------------------------------------------------------------------%
    \experiment{1}{
        \noindent\textbf{Write C programs by using Array data structure for the following problem domains:}
        \begin{enumerate}[label=(\alph*)]
            \item Create an array of integer with size $n$.
            Return the difference between the largest and the smallest value inside that array.
            \item Initializes an array with ten random integers and then prints four lines of output, containing: Every
            element at an even index, Every odd element, All elements in reverse order, Only the first and last element.
            \item Consider an integer array of size 5 and display the following: Sum of all the elements, Sum of
            alternate elements in the array, and second-highest element in the array.
        \end{enumerate}\vspace{\onehalfspacing}%
    }\\%
    \begin{code}
        {Program (a)}{c}
#include <stdio.h>

int main() {
    int arr[] = {1, 2, 3, 4, 5, 6, 7, 8, 9, 10};
    int n = (int) (sizeof(arr) / sizeof(int));

    int min = 0, max = 0;

    for (int i = 0; i < n; i++) {
        if (arr[i] < min)
            min = arr[i];
        if (arr[i] > max)
            max = arr[i];
    }

    printf("Minimum = %d; Maximum = %d; Difference = %d", min, max, max-min);

    return 0;
}
    \end{code}%
    \begin{code}
        {Output (a)}{text}
Minimum = 0; Maximum = 10; Difference = 10
    \end{code}
    %------------------------------------------------------------------------------------------------------------------%
    \begin{code}
        {Program (b)}{c}
#include <stdio.h>

int main() {
    int arr[] = {1, 2, 3, 4, 5, 6, 7, 8, 9, 10};
    int n = (int) (sizeof(arr) / sizeof(int));

    printf("Elements at an even index: ");
    for (int i = 0; i < n; i += 2) {
        printf("%d, ", arr[i]);
    }
    printf("\n");

    printf("Odd numbers: ");
    for (int i = 0; i < n; i++) {
        if (arr[i] % 2 != 0)
            printf("%d, ", arr[i]);
    }
    printf("\n");

    printf("Reverse order: ");
    for (int i = n - 1; i >= 0; i--) {
        printf("%d, ", arr[i]);
    }
    printf("\n");

    printf("First: %d; Last: %d;\n", arr[0], arr[n - 1]);

    return 0;
}
    \end{code}%
    \begin{code}
        {Output (b)}{text}
Elements at an even index: 1, 3, 5, 7, 9,
Odd numbers: 1, 3, 5, 7, 9,
Reverse order: 10, 9, 8, 7, 6, 5, 4, 3, 2, 1,
First: 1; Last: 10;
    \end{code}
    %------------------------------------------------------------------------------------------------------------------%
    \begin{code}
        {Program (c)}{c}
    \end{code}%
    \begin{code}
        {Output (c)}{text}
Hello, World!
    \end{code}
    \vspace*{10pt}

    % ---------------------------------------------------------------------------------------------------------------- %
\end{document}