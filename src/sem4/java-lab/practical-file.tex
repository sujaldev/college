%! Author = Sujal Singh
%! Date = 1/17/25

% Preamble
%! suppress = FileNotFound
\documentclass[11pt]{ipu-ai}
\doctitle{OOPS With Java Lab (ARI256)}

% Packages
\usepackage{amsmath}
\usepackage{amssymb}
\usepackage[T1]{fontenc}
\usepackage[
    pdftitle={OOPS With Java Lab Practical File},
    pdfsubject={OOPS With Java Lab Practical File},
    pdfauthor={Sujal Singh},
    pdfdisplaydoctitle,
    hidelinks,
]{hyperref}
\usepackage{tabularray}
\usepackage{tikz}
\usepackage[indLines]{algpseudocodex}
\usepackage{enumitem}


\renewcommand{\experiment}[2]{%
%    \newpage%
    \begin{center}%
        \textbf{\Huge Experiment--#1}\\[30pt]%
    \end{center}%
    \begin{tabularsection}{Aim}%
        ~\\#2\\%
    \end{tabularsection}}

% Document
\begin{document}
    \maketitle
    \pagenumbering{gobble}
%\pagestyle{empty}
\begin{center}
    \textbf{\huge Index} \\
    \begin{longtblr}{rows={35pt},colspec={|Q[m,c]|Q[12,m]|Q[2,c]|},hlines,vlines,
        cell{2-Z}{1} = {cmd=\textbf{\the\numexpr\arabic{rownum}-1}.},
        cell{1}{2}={c}}
        \textbf{No.} & \textbf{Question} & \textbf{Remarks} \\
        &%
        \newline
        \noindent\textbf{Write C programs by using Array data structure for the following problem domains:}
        \begin{enumerate}[label=(\alph*)]
            \item Create an array of integer with size $n$.
            Return the difference between the largest and the smallest value inside that array.
            \item Initializes an array with ten random integers and then prints four lines of output, containing: Every
            element at an even index, Every odd element, All elements in reverse order, Only the first and last element.
            \item Consider an integer array of size 5 and display the following: Sum of all the elements, Sum of
            alternate elements in the array, and second-highest element in the array.
        \end{enumerate}
        & \\
        &%
        \newline
        \noindent\textbf{Write a program to create a singly linked list of n nodes and perform:}
        \begin{enumerate}[label=(\alph*)]
            \item Insertion at the beginning.
            \item Insertion at the end.
            \item Insertion at a specific location.
            \item Deletion at the beginning.
            \item Deletion at the end.
            \item Deletion At a specific location.
        \end{enumerate}
        & \\
        &%
        \newline
        \noindent\textbf{Write a program to create a doubly linked list of $n$ nodes and perform:}
        \begin{enumerate}[label=(\alph*)]
            \item Insertion at the beginning.
            \item Insertion at the end.
            \item Insertion at a specific location.
            \item Deletion at the beginning.
            \item Deletion at the end.
            \item Deletion At a specific location.
        \end{enumerate}
        & \\
        &%
        \newline
        \noindent\textbf{Write a program to create a singly circular and doubly linked list of $n$ nodes and perform:}
        \begin{enumerate}[label=(\alph*)]
            \item Insertion at the beginning.
            \item Insertion at the end.
            \item Insertion at a specific location.
            \item Deletion at the beginning.
            \item Deletion at the end.
            \item Deletion At a specific location.
        \end{enumerate}
        & \\
        &%
        Write a program to implement stack using arrays and linked lists.
        & \\
        &%
        Write a program to reverse a sentence/string using stack.
        & \\
        &%
        Write a program to check for balanced parenthesis in a given expression.
        & \\
        &%
        Write a program to convert infix expression to prefix and postfix expression.
        & \\
        &%
        Write a program to implement Linear Queue using Array and Linked Lists.
        & \\
        &%
        Write a program to implement Circular Queue using Array and Linked Lists.
        & \\
        &%
        Write a program to implement Doubly Ended Queue using Array and Linked Lists.
        & \\
        &%
        Write a Program to implement Binary Search Tree operations.
        & \\
        &%
        Write a program to implement Bubble Sort, Selection Sort, Heap Sort, Quick Sort, Merge Sort and Insertion Sort algorithm.
        & \\
        &%
        \newline
        \noindent\textbf{Write C Programs by using Graph data structure for the following problem domains:}
        \begin{enumerate}[label=(\alph*)]
            \item Graph Traversal: BFS
            \item Graph Traversal: DFS
        \end{enumerate}
        & \\
    \end{longtblr}
\end{center}
\newpage%
\pagenumbering{arabic}%

    % ---------------------------------------------------------------------------------------------------------------- %

    %------------------------------------------------------------------------------------------------------------------%
    %------------------------------------------------------------------------------------------------------------------%
    \experiment{1}{Writing a program, compiling and running using command line in Java.}\\%
    \begin{code}
        {Program}{java}
public class q1 {
    public static void main(String[] args) {
        System.out.println("Hello, World!");
    }
}
    \end{code}%
    \begin{code}
        {Output}{text}
Hello, World!
    \end{code}
    \vspace*{10pt}

    % ---------------------------------------------------------------------------------------------------------------- %
    % ---------------------------------------------------------------------------------------------------------------- %

    \experiment{2(a)}{Write a program to show if the given number is a prime number using for loop.}%
    \begin{code}
        {Program}{java}
public class q2a {
    public static void main(String[] args) {
        int n = 67;

        for (int i = 2; i <= (n / 2); i++) {
            if (n % i == 0) {
                System.out.println("Composite number.");
                return;
            }
        }
        System.out.println("Prime number.");
    }
}
    \end{code}%
    \begin{code}
        {Output}{text}
Prime number.
    \end{code}
    \newpage%

    % ---------------------------------------------------------------------------------------------------------------- %
    % ---------------------------------------------------------------------------------------------------------------- %

    \experiment{2(b)}{Write a program to show if the given number is a prime number using while loop.}%
    \begin{code}
        {Program}{java}
public class q2b {
    public static void main(String[] args) {
        int n = 67;

        int i = 2;
        while (i <= (n / 2)) {
            if (n % i == 0) {
                System.out.println("Composite number.");
                return;
            }
            i += 1;
        }
        System.out.println("Prime number.");
    }
}
    \end{code}%
    \begin{code}
        {Output}{text}
Prime number.
    \end{code}

    % ---------------------------------------------------------------------------------------------------------------- %

\end{document}