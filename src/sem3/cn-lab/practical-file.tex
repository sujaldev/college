%! Author = Sujal Singh
%! Date = 12/3/24

% Preamble
%! suppress = FileNotFound
\documentclass[11pt]{ipu-python}
\doctitle{Computer Networks Lab (ARI257)}

% Packages
\usepackage{amsmath}
\usepackage[
    pdftitle={Computer Networks Lab Practical File},
    pdfsubject={Computer Networks Lab Practical File},
    pdfauthor={Sujal Singh},
    pdfdisplaydoctitle,
    hidelinks,
]{hyperref}

% Document
\begin{document}
    \maketitle

    %------------------------------------------------------------------------------------------------------------------%
    %------------------------------------------------------------------------------------------------------------------%

    \pagenumbering{gobble}
    \pagestyle{empty}
    \begin{center}
        \textbf{\huge Index} \\[20pt]
        \begin{tblr}{rows={50pt},colspec={|Q[m,c]|Q[12,m]|Q[3,c]|},hlines,vlines,
            cell{2-Z}{1} = {cmd=\textbf{\the\numexpr\arabic{rownum}-1}.},
            cell{1}{2}={c}}
            \textbf{No.} & \textbf{Experiment} & \textbf{Remarks} \\
            &%
            To understand the basics of computer networks, including types of networks, network topologies, and types of
            cables.
            & \\
            &%
            To understand IP addresses, including versions, unicast, multicast, broadcast, classes of IP addresses,
            subnetting, and supernetting.
            & \\
            &%
            To study various network devices including repeaters, switches, hubs, routers, and gateways.
            & \\
            &%
            To configure network setting and learn a few networking commands.
            & \\
            &%
            To learn about the colour coding of straight-through, crossover and roll-over cables.
            & \\
            &%
            To create a LAN using crossover-cable, hub and switch.
            & \\
            &%
            To connect and configure a switch and create a VLAN\@.
            & \\
            &%
            To connect and configure a router in a network using Cisco Packet Tracer.
            & \\
            &%
            To connect two different networks through a router.
            & \\
            &%
            To connect two different networks through routers using RIP (Routing Information Protocol) in Cisco Packet
            Tracer.
            & \\
            &%
            To connect two different routers using static routing protocols in Cisco Packet Tracer.
            & \\
        \end{tblr}
    \end{center}
    \newpage%
    \pagenumbering{arabic}%
\end{document}