%! Authors = Dhruv Grover, Pranav Bisht, Sujal Singh
%! Date = 11/20/23

% Preamble
\documentclass[11pt]{beamer}
\logo{ipu-logo}
\title{Dynamic Memory Allocation in C}
\author{Dhruv Grover \(|\) Pranav Bisht \(|\) Sujal Singh}
\date{}

\usetheme{Madrid}
\setbeamertemplate{frametitle continuation}[from second][]

% Packages
\usepackage{amsmath}
\usepackage{blkarray}
\usepackage[outputdir=./build]{minted}
\usepackage{tcolorbox}

% Document
\begin{document}
    \maketitle

    \begin{frame}{The What}
        \textbf{\large What is dynamic memory allocation?}\\[10pt]
        \begin{itemize}
            \item Dynamic memory allocation is the process of allocating memory at \textbf{runtime}.
            \item Dynamically allocated memory is allocated in a pool of memory called the \textbf{heap} (it has nothing
            to do with the heap data structure) which can grow as required.
            \item The amount of space required does not have to be known by the compiler in advance.
            \item To read or write to dynamically allocated memory, \textbf{pointers} must be used.
            \item Allocated block of memory can also be \textbf{resized} or \textbf{freed} after allocation at runtime.
        \end{itemize}
    \end{frame}

    \begin{frame}[allowframebreaks]{The Why}
        \textbf{Why do we need dynamic memory?}\\[10pt]
        Statically allocated memory has some limitations, such as:\\[5pt]
        \begin{itemize}
            \item Exact size and type of storage must be known at \textbf{compile time}.
            \item Since statically allocated memory cannot be resized at runtime
            \begin{itemize}
                \item It can cause \textbf{wastage of memory} if the amount of memory used is less than the memory
                allocated at compile time.
                \item It cannot handle the scenario wherein more memory is required at runtime than initially declared
                during compilation.
            \end{itemize}
        \end{itemize}
        And so, a need for another type of memory allocation arises.

        \framebreak

        Dynamically allocated memory solves these problems:\\[5pt]

        \begin{itemize}
            \item Unlike static memory, dynamic memory does not require the type and size to be declared during
            compilation.
            \item If the amount of space required in a dynamically allocated block increase or reduces, the allocated
            block of memory can be re-allocated to match the new space requirements.\ This decreases wastage of memory.
        \end{itemize}

        \begin{block}{Note}
            It is important to note that using dynamically allocated memory doesn't automatically solve undefined
            behaviour, the allocated memory still has to be manually reallocated if the required amount of memory
            changes.
        \end{block}
    \end{frame}

    \begin{frame}[allowframebreaks,containsverbatim]{The How}
        \textbf{The Standard Library (\texttt{<stdlib.h>})}\\[10pt]

        The standard library in C provides some functions that handle dynamic memory allocation, these are:\\[5pt]
        \begin{itemize}
            \item \textbf{\texttt{malloc()}}
            \item \textbf{\texttt{calloc()}}
            \item \textbf{\texttt{realloc()}}
            \item \textbf{\texttt{free()}}
        \end{itemize}
        \\[10pt]
        Let us look at these in depth individually.

        \framebreak

        \textbf{\texttt{\large malloc()}}\\[10pt]

        \begin{itemize}
            \item Short for ``memory allocate''.
            \item It is used to dynamically allocate a block of memory of specified size.
            \item Returns a void pointer to the starting address of the allocated block of memory.
            \item It can also return \texttt{NULL} if it is unable to find the requested space in memory.
            \item Example:
            \begin{tcolorbox}
                \begin{minted}{c}
int *ptr = (int*) malloc(2 * sizeof(int));
                \end{minted}
            \end{tcolorbox}
        \end{itemize}


        \framebreak

        \textbf{\texttt{\large calloc()}}\\[10pt]

        \begin{itemize}
            \item Short for ``clear allocate''.
            \item It is similar to \texttt{malloc()} except that it ensures that the allocated block of memory
            contains zero initialized bytes.
            \item Example:
            \begin{tcolorbox}
                \begin{minted}{c}
int *ptr = (*int) calloc(2, sizeof(int));
                \end{minted}
            \end{tcolorbox}
        \end{itemize}

        \framebreak

        \textbf{\texttt{\large realloc()}}\\[10pt]

        \begin{itemize}
            \item Short for ``re-allocate''.
            \item It is used to resize an already allocated block of memory without losing the old data.
            \item If \texttt{realloc()} can't find the space in memory where the currently allocated block exists, it
            will copy it to a new location where the requested amount of space is available.
            contains zero initialized bytes.
            \item Example:
            \begin{tcolorbox}
                \begin{minted}{c}
int *ptr = (int*) realloc(ptr, 3 * sizeof(int));
                \end{minted}
            \end{tcolorbox}
        \end{itemize}

        \framebreak
        \textbf{\texttt{\large free()}}
        \begin{itemize}
            \item It is used to free up memory that was dynamically allocated by using \texttt{malloc()} or
            \texttt{calloc()}
            \item It helps in preventing wastage of memory.
            \item Example:
            \begin{tcolorbox}
                \begin{minted}[fontsize=\small]{c}
#include <stdlib.h>

void main() {
    // Allocate some memory using malloc
    int *ptr = (int*) malloc(2 * sizeof(int));

    // Allocate some memory using malloc
    free(ptr);
}
                \end{minted}
            \end{tcolorbox}
        \end{itemize}

    \end{frame}

    \begin{frame}{End}
        \begin{center}
            \textbf{\Huge The End}
        \end{center}
    \end{frame}
\end{document}